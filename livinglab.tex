Building on the ethos and practice of the Edinburgh Living Lab… We wanted to explore the possibility for engaging users more in designing applications for the Internet of Things. Meeting room occupancy was an identified problem in Argyle House that affected most of the occupants, and therefore it seemed to be an appropriate topic for exploring user engagement.

As we were preparing for the rollout of an experimental IoT network on the University campus and eventually across the city, we wanted to anticipate possible adverse user reactions and design our projects to preclude those. We were aware of IoT projects on other University campuses and in other places that had to be stopped because of complaints about invasion of privacy and being monitored.
We were aware of some of the mistakes those projects had made - e.g. not adequately informing users - but rather than only try to correct others’ mistakes, we wanted to take a proactive approach to understanding user perceptions and values and we wanted to identify how those could be incorporated into the design of the project from the outset.

Our original concept was to anticipate human interface issues and identify principles for design for social acceptance / acceptability. We expected to do this by a) identifying points of concern with the existing monitoring project, b) exploring whether the issues being monitored were really the most important building concerns for users, and c) understanding user perceptions of the opportunities and pitfalls of monitoring. ...improving communication...

On top of this, we wanted to explore theoretically - and eventually in practice, given the opportunity to carry out a second phase of the research - how we could engage users in the design of workplace monitoring. This was based on an assumption that the co-design process would improve awareness and acceptability. This would happen as a result of some people being directly involved in the co-design, by creating a general perception that the project had key user input, and by ensuring that knowledge and understanding of the project was not limited to management but was shared across staff members who would in turn share information with their colleagues.

We thought as well that, besides increasing awareness and acceptability, user participation in the design process would lead to new ideas for applications of the Internet of Things that addressed users’ perceived problems (which were not necessarily the same as building managers’ perceived problems) and potentially lead to more innovative or creative approaches to deploying IoT for the purpose of improving staff (user) experience in the workplace. ...value of user-centred design...
