The office monitoring project aligned well with one of the strategic
objectives of the University of Edinburgh's IoT Initiative, namely
using IoT technologies to improve internal service delivery and
operations. A second component of the strategy proposes using the city
--- and by extension, the University campus --- as a Living Lab for
IoT proof-of-concept experiments \cite{IoT-Strategy}. We interpret
this to mean that a comprehensive approach to deploying IoT
technologies must take into account the human dimension of those
deployments \cite{Shin-2017-UTIO}.  People whose lives are
impacted by IoT applications should have input into how those
applications are designed, used and acted upon. Meeting room occupancy
had been identified as an issue that affected most of the occupants in
the office block that we studied, and therefore seemed to be an
appropriate topic for exploring user engagement with living lab
methods. 

The office monitoring project was regarded as a pilot for
developing a wider range of IoT-based interventions across the campus,
and these would inevitably involve the student population to a greater
or lesser degree. And as just indicated, thr University's initiative
envisages deploying IoT applications more broadly across the city as a
whole. Consequently, we felt it was important to anticipate possible
adverse reactions from people of all walks of life and to design our
projects to preclude or minimise those.  

In order to identify principles of design to support social
acceptance, our user study started with the following goals: (i)
identifying points of concern with the existing monitoring project;
(ii) exploring whether the issues being monitored were really the most
important building concerns for users; and (iii) understanding user
perceptions of the opportunities and pitfalls of monitoring.  We also
hoped to gauge the feasibility of engaging users from the outset in
monitoring projects such as the one described in
Section~\ref{sec:pilot}. Could adoption of a co-design process improve
awareness and acceptability? In principle, this should happen as a
result of office users being directly involved in the monitoring
design, by creating a general perception that the project had key user
input, and by ensuring that knowledge and understanding of the project
was not limited to management but was shared across staff members who
would in turn share information with their colleagues.

Besides increasing awareness and acceptability, it is likely that user
participation in the design process would lead to new ideas for IoT
applications that addressed users’ perceived problems (which were not
necessarily the same as those of the building managers) and
potentially lead to more innovative or creative approaches to
deploying IoT for the purpose of improving staff experience in
the workplace.  

%%% Local Variables: ***
%%% mode:latex ***
%%% coding: utf-8 ***
%%% TeX-engine: xetex ***
%%% TeX-master: "main.tex"  ***
%%% End: ***