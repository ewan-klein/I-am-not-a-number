The Argyle House monitoring pilot aligns well with one of the strategic objectives of the University of Edinburgh's IoT Initiative, namely using IoT technogies to improve internal service delivery and operations. However, a second objective introduces the idea of using the city --- and by extension, the University campus --- as a Living Lab for IoT proof of concept experiments \cite{IoT-Strategy}. We interpret this to mean that a comprehensive approach to deploying IoT technologies must take into account the human dimension of those deployments. Ideally, people whose lives are impacted by IoT applications should have input into how those applications are designed, used and acted upon. Meeting room occupancy was an identified problem in Argyle House that affected most of the occupants, and therefore it seemed to be an appropriate topic for exploring user engagement.

The Argyle House project was in part regarded as a pilot for exploring a wider range of IoT-based interventions across the campus, and these would inevitably involve the student population to a greater or lesser degree. Indeed, in the longer term, we expect to rollout IoT applications more broadly across the city as a whole. Consequently, we felt it was important to anticipate possible adverse reactions from people of all walks of life and design our projects to preclude or minimise those. 

In order to identify principles of design to support social acceptance, our initial project was based on the following goals:
\begin{itemize}
\item identifying points of concern with the existing monitoring project,
\item exploring whether the issues being monitored were really the most important building concerns for users, and
\item understanding user perceptions of the opportunities and pitfalls of monitoring.
\end{itemize}

While we were aware of some of the dangers of not adequately informing users, we wanted to do more than avoid mistakes by adopting a proactive approach to understanding user perceptions and values. How we could engage users in the design of workplace monitoring from the outset? This was based on an assumption that the co-design process would improve awareness and acceptability. This would happen as a result of some people being directly involved in the co-design, by creating a general perception that the project had key user input, and by ensuring that knowledge and understanding of the project was not limited to management but was shared across staff members who would in turn share information with their colleagues.

We thought as well that, besides increasing awareness and acceptability, user participation in the design process would lead to new ideas for IoT applicationsthat addressed users’ perceived problems (which were not necessarily the same as building managers’ perceived problems) and potentially lead to more innovative or creative approaches to deploying IoT for the purpose of improving staff (user) experience in the workplace. 

%%% Local Variables: ***
%%% mode:latex ***
%%% TeX-master: "main.tex"  ***
%%% End: ***