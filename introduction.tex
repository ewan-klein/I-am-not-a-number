Lit review (of sorts)
IoT and resource efficiency / user experience

Early narratives of the Internet of Things envisioned a ‘smart planet’ where everything from ___ to ____ functioned more efficiently and effectively achieved its purpose. Sensors and other forms of data collection would allow us to know exactly what was happening where; advanced data analytics would help us to understand and integrate data collected from different sensors and other sources; and remote management of devices and systems - and eventually automated interaction between devices - would allow those systems to adapt and change in real-time, immediately responding to incoming information and adjusting to optimise for _______.
This narrative was deeply embedded in the concept of the ‘smart city’
…

For example…

Types and purposes of IoT monitoring

As this narrative developed and practical applications were tested,
questions began to arise about the foundation and assumptions of its
vision. Where are the people in the vision of the smart city (Mara
Balestrini)? The smart building? And who and what are we optimising
for? What biases are we building into our algorithms and who might
that affect?

Living Lab and human dimensions of IoT

Citizen sensing, smart citizen, experimentation in the wild - testing
what are the implications, effects, will this achieve what we
envisioned, how will people respond, how will people interact with it,
etc.

Technocratic vision is also embedded in the ‘smart building’ ...

For example...

IoT monitoring in offices - goals, values, examples, etc. - monitoring
of spaces for use (usually done with other types of data analytics?) -
monitoring of spaces for comfort

Massive amount of work in the energy sector on communication of information collected through IoT to influence behaviour change
IoT and user experience

Despite movement toward co-design, much research and development of IoT products and services (and smart buildings) focuses on user experience within the relatively narrow space of interaction design.
Also on improving services...

User engagement (or at least user consideration) to improve awareness
and acceptability

As IoT has expanded and more and more data is being collected from
people - and they are becoming more and more aware of it - big
questions around awareness, acceptability, privacy and trust have
arisen. It is now acknowledged that these issues must be addressed and
various studies are recommending different ways to do that. 
 
Co-design of the use of IoT in buildings
Has happened with citizen sensing to some degree (Mara Balestrini and
frogs / damp in Bristol - was an open-ended question to ask people
what problems they had that they could work on with IoT) but not many
examples in buildings

Perceptions of being monitored (office monitoring)

Couldn’t find much work in this space…

The future of work

The workplace is an interesting place to explore perceptions and co-design of the use of IoT because...


%%% Local Variables: ***
%%% mode:latex ***
%%% TeX-master: "main.tex"  ***
%%% End: ***