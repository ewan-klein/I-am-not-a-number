Early narratives of the Internet of Things (IoT)envisioned a ‘smart planet’
where sensors and other forms of data
collection would tell us exactly what was happening where;
advanced data analytics would help us to understand and integrate data
collected from different sensors and other sources; and remote
management of devices and systems --- and eventually automated
interaction between devices --- would allow those systems to adapt and
change in real-time, immediately responding to incoming information
and providing actionable insights for managers and decision makers.  

Much of the interest in IoT has focussed on the notion of a 'smart
environment', which in turn can be broken down into various
application domains, such as smart home, smart office, smart retail,
smart city, smart agriculture/forest, smart water and smart transport \cite{Gubbi-2013-IOT}.
The vision of the smart city as generated considerable discussionm and
as noted by \cite{Hollands-2015-CIIT}, 

\begin{quote}
  the smart city is currently being constructed as the solution to
  many urban problems, including crime, traffic congestion, inefficient
  services and economic stagnation, promising prosperity and healthy
  lifestyles for all. In short, the smart city symbolises a new kind
  of technology-led urban utopia\ldots
\end{quote}

However, this vision has attracted an increasing amount of criticism,
particularly on grounds that the agenda has primarily been set by the
interests of global business \cite{Hollands-2015-CIIT}, and that it
takes a top-down perspective with smart technology as a starting
point, rather than starting from ``ordinary urban places, knowledges
and needs'' \cite{Mcfarlane-2017-OASC}. The counter-narrative of
\term{Smart Citizens} argues that ordinary people should be at the
centre of smart urbanism: ``citizens can, and should, play a leading
role in conceiving, designing, building, maintaining our cities of the
future'' \cite{Hemment-2013-SC,Hemment-2016-HTDU}.
Digital technologies, including low-cost sensing devices, make it
possible for citizens themselves to help co-create smart urbanism by
directly participating in the IoT ecosystem \cite{Balestrini-2017-OCTT}.

As an environment, the office shares characteristics with both the
city and the home. Motivations for creating a `Smart Office' have
tended to share the values which are generally provided for IoT
deployments, such as improving operations, optimizing assets, enhancing
services, and providing security
\cite{Heidt-2016-PGFT,Gaur-2015-SCAA,Gubbi-2013-IOT}. However, unlike
the case of smart urbanism, there seems to have been little
critical debate about developing a person-centred vision of IoT in the
office --- there is no narrative of `Smart Office Workers'.  In this
paper, we describe a small study of attitudes and responses by university office
staff to the pilot deployment of networked sensors for measuring
occupancy and environmental conditions in an office block. 
 
 

% Citizen sensing, smart citizen, experimentation in the wild - testing
% what are the implications, effects, will this achieve what we
% envisioned, how will people respond, how will people interact with it,
% etc.

% Technocratic vision is also embedded in the ‘smart building’ ...

% For example...

% IoT monitoring in offices - goals, values, examples, etc. - monitoring
% of spaces for use (usually done with other types of data analytics?) -
% monitoring of spaces for comfort

% Massive amount of work in the energy sector on communication of information collected through IoT to influence behaviour change
% IoT and user experience

% Despite movement toward co-design, much research and development of IoT products and services (and smart buildings) focuses on user experience within the relatively narrow space of interaction design.
% Also on improving services...

% User engagement (or at least user consideration) to improve awareness
% and acceptability

% As IoT has expanded and more and more data is being collected from
% people - and they are becoming more and more aware of it - big
% questions around awareness, acceptability, privacy and trust have
% arisen. It is now acknowledged that these issues must be addressed and
% various studies are recommending different ways to do that. 
 
% Co-design of the use of IoT in buildings
% Has happened with citizen sensing to some degree (Mara Balestrini and
% frogs / damp in Bristol - was an open-ended question to ask people
% what problems they had that they could work on with IoT) but not many
% examples in buildings

% Perceptions of being monitored (office monitoring)

% Couldn’t find much work in this space…

% The future of work

% The workplace is an interesting place to explore perceptions and co-design of the use of IoT because...


%%% Local Variables: ***
%%% mode:latex ***
%%% TeX-master: "main.tex"  ***
%%% End: ***