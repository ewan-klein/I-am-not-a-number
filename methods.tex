
\subsection{Study Design}
\label{sec:study-design}


The main purpose of the study was to understand people’s qualitative perceptions of monitoring in the workplace. We used semi-structured interviews and focus groups to explore individual perspectives.
Interviews and focus groups were structured around three main questions:
What are occupants’ perceptions of occupancy monitoring in their workplace?
How would occupants’ perceptions of occupancy monitoring change (or not) if they were engaged in a process of identifying key building issues that monitoring could help resolve and exploring scenarios of how that could happen?

What do occupants want from monitoring?

We adapted the interviews (but not the focus groups) slightly from the original plan to include a simple data visualisation. We asked participants to ‘interpret’ the data visualisation and what impact they thought making the monitoring data public through a visualisation might have on people’s perceptions of the monitoring and their ability and interest to engage with the monitoring project.

\subsection{Participant Recruitment}
\label{sec:recruitment}

We recruited staff in Information Services who were based in Argyle House, where the monitoring pilot was installed. An email from an IT director was sent to all staff inviting them to participate (see Appendix). A separate email was sent to managers in divisions to highlight the project and ask them to encourage their staff to engage. Due to low initial response, additional reminders were sent out.

We confirmed 9 interview participants and 6 focus group participants, divided into a group of 2 and a group of 4 based on availability. One of the interviewees was the head of facilities management; some of the others had heard of the monitoring project and one had seen it come through a security review, but otherwise none of them were directly involved with it. Four interview participants were female and five were male. Four focus group participants were female and two were male. We did not ask people’s ages, but a general estimate is that most participants were between 30 and 50 years of age.

\subsection{Data Analysis}
\label{sec:data-analysis}


Interviews were recorded, and the transcriptions and hand-written notes were used to analyse the data.
