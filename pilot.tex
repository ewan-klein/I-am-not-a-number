As part of a broad plan for estates management by the University of
Edinburgh, several hundred staff from an operational
services department, previously scattered across multiple sites, were
rehoused in 2016/17 in open-plan offices within refurbished shared
commercial premises. To complement the open-plan spaces, the ground
floor of the premises was completely remodelled to contain an
assortment of large and small meeting rooms. 

In a separate initiative that had been building momentum throughout
2016, the University of Edinburgh established an Internet of Things
network based on LoRaWAN, a secure long-range and low-power radio
communications technology using the unlicensed spectrum for message
broadcast by small battery-powered sensor devices. It was decided to
help test the network by carrying out a pilot project to collect
information about occupancy of the meeting rooms and environmental
conditions in the office spaces more generally. This information was
intended to help address two questions of concern to the University’s
building managers: (i) was the configuration of meeting spaces
appropriate for the needs of occupants, and (ii) were the
environmental conditions, particularly in terms of heating and
ventilation, satisfactory in the open plan offices? 

The pilot project deployed multiple devices that were networked using
a combination of Bluetooth and LoRaWAN protocols. First, in one of the small
meeting rooms, Estimote beacons with auxiliary sensors were attached
to chairs.%
\footnote{
 In addition to using the Estimote beacon's inbuilt accelerometer, its
 external GPIO pin was connected to a force-sensitive resistor in the
 cushion. This combination allowed us to measure whether the chair was moved and
 whether it was occupied within each two minute timeframe. The
 beacons were configured to transmit data over Bluetooth at least once a
 second. The data packets were collected by
 a Pycom LoPy sensor hub,  which sent a summary message over LoRaWAN to a
 central server every two minutes. } 
An additional device containing an accelerometer and light-level
sensor was placed
on the door, thus capturing
when the door was moved and whether the light was on in the
room. Second, sensors for light level, temperature, atmospheric pressure
and relative humidity were placed near desks across one wing of an
open-plan area.

Before installing the sensors, we carried out a Privacy Impact
Assessment (PIA) in which we were careful to demonstrate that no
personal data would be captured by the sensors.\footnote{PIAs are
  becoming a critical part of the engagement process as part of new
  data protection legislation such as the General Data Protection
  Regulation \cite{GDPR}.} Information about the pilot was
communicated in two ways: by sending an initial email announcement to
all building users, and by attaching a notice explaining the project
to the wall of the monitored meeting room. Figure~\ref{fig:poster}
illustrates the poster, which includes illustrations of the Estimote
beacons used as sensors.
\begin{figure}
  \centering
  \includegraphics[scale=0.35]{images/info-poster}
  \caption{The information poster that
    was placed prominently on the wall of the meeting room where
    sensors had been installed.}
  \label{fig:poster}
\end{figure}
As we discuss in section~\ref{sec:results}, subsequent investigation
cast into doubt whether these communication steps were adequate in providing
building users with an appropriate level of insight into the
monitoring work.

%%% Local Variables: ***
%%% mode: latex ***
%%% coding: utf-8 ***
%%% TeX-engine: xetex ***
%%% TeX-master: "main.tex"  ***
%%% End: ***