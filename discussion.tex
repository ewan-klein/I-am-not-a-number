‘Standard’ → framework
Purpose
Transparency
Participation

“Awareness, acceptability, trust”
Participants’ responses and the types of people that we identified suggest that there is work to be done in a number of areas if we are to design IoT experiments based on an appropriate understanding and anticipation of how people will respond to them.

With top-down - or management or technophile-driven - experiments, there is likely to be some disjunct between the vision of what could or should be achieved (what is promised) and what is actually delivered or accomplished. … “over-collection” of data
…
We might define a ‘successful’ experiment as one that leads to awareness, acceptability and trust. Or we could define it as an experiment that either a) accomplishes the experimenter’s goals without adverse reactions or b) involves participants in co-creating the goals and leads to a positive experience that IoT has ‘accomplished’ something for the participants. 

We propose three principles for awareness, acceptability and trust.

Purpose: Monitoring should be a time-bounded activity with a clearly defined purpose. Individuals who might be affected by the monitoring should be informed about the what, why and how of decisions made based on the data collected. Overmonitoring by “well-meaning technophiles” should be avoided when simpler interventions could achieve the desired goal.

Transparency: Anyone affected by the monitoring should be informed about what data is being collected and should be able to access an easily interpretable explanation of this, for example through data visualisation. Physical signage in monitored areas is more effective than email communication. If possible, some version of the raw data should be accessible to anyone interested in exploring it and comparing their interpretations to that of the decision-makers.

Participation: In the case of workplace monitoring, there is an opportunity and a need to move beyond the narratives of resource efficiency, maximising productivity and incentivising specific behaviours. Trust in management decision-making has been compromised by the use of data to justify cost-cutting and provide only a minimum acceptable level of facilities and resources.

New narratives built around trust and valuing individuals can be created by involving employees in identifying the issues that most affect their performance, comfort, health and well-being and determining how and whether monitoring could be used to address these issues effectively. The engagement process can be structured to offer employees freedom, creativity, and a sense of agency.

The narrative of the value of the Internet of Things to maximise resource efficiency and improve user experience is widely embedded and offers a seemingly easy ‘quick win’ application. However, how such an application is designed and delivered can influence its acceptability and either limit or expand its potential to create shared value. These three guidelines can help to overcome skepticism and ensure that IoT monitoring projects incorporate a broader set of values and beneficiaries.
