\subsection{Perceptions of monitoring}
\label{sec:perc-monit}
Among the many different and mixed responses to the idea of IoT
monitoring in the workplace, most people seemed to fall more or less
into one of the following categories:
\begin{enumerate}
\item \textit{I'm not so bothered} (FG1, P2), (I4) --- I'm already being monitored in so many ways
  that are so much more personal e.g., CCTV in the building, mobile
  phone apps, etc. 
\item \textit{Data can help} (I1, I3, I5, I6) --- There are so many ways that we could use
  data like this to improve our workplace experience and the work that we do. We can create a lot of value and benefit if we use this technology well. 
\item \textit{Show me the change} (I2), FG2, P3) --- Is all this monitoring really necessary?
  Can I see what data you’re collecting? Will the data actually be
  used to inform a change? How will I know if it is? Might there be
  easier or simpler ways to achieve the necessary change? 
\end{enumerate}
The first category represents someone who already willingly shared a
lot of personal data or was completely aware of how much personal data
was already being collected from him/her with or without transparency
and well-informed consent e.g., through various apps. The second
category reflected a person with previous experience of using sensors
or data analytics in their work. They were not likely to react
negatively to the monitoring and were keen to contribute ideas and
suggestions about more monitoring options and how they could be used
to improve user experience and services generally. The third response
was found to some degree in many of the conversations but was
particularly pronounced in a few cases, which represented people with
the most potential to view IoT initiatives skeptically or negatively.

\subsection{Challenges}
\label{sec:challenges}

The responses of our study participants indicate that there are number
of challenges to be addressed in engaging with users when deploying
IoT applications in offices and other places of work.

One of the most striking features of the participants' responses was
the extent to which they felt underinformed about the nature and goals
of the office monitoring project. Although the initial measures that
we took (email, information poster) in communicating with staff about
the project 'ticked the box' in terms of standard approaches to data
protection compliance, it fell short in engaging their attention and
interest to properly understand the what, how and why of the data
collection. 

Privacy was not originally a primary concern of our study. Nevertheless,
reflections on privacy and surveillance surfaced frequently in the
comments of the study participants.

We can frame some of these concerns in terms of \term{information
  spaces} \cite{Jiang-2002-MPCI} in the sense of ``a way to organize
information, resources, and services around important privacy-relevant
context factors.\ldots\ A \term{boundary}---physical, social, or
activity-based---delimits an information space.'' In the monitoring
experiment, physical boundaries were clearly important in determining
information-spaces. We chose to monitor a meeting room specifically
because it was a common space, not linked to any specific individuals,
and therefore less likely to trigger privacy concerns. Conversely, the
desks that people occupied in the open plan offices defined much more
personal information spaces, albeit not demarcated by such clearcut
physical boundaries. It is clear from comments in
Section~\ref{sec:results} that many participants saw monitoring of
their desks as a source of concern. While this was not being
considered as an extension of the meeting room monitoring, many people
viewed an initial monitoring project as a potential ‘foot in the door’
to expand and extend monitoring activities.

The chairs in the monitored meeting room were somewhat less easy to
classify in this scheme. Again, they can be regarded as information
spaces, given that sensors were attached to them. And although they
are only transiently associated with any one individual (i.e., for the
duration of a meeting), the study results show that several participants were
sensitive about the physical closeness of the
sensors. \cite{Ohara-2016-TSVP} proposes to characterise a breach of privacy as
the state which arises when one of `my' boundaries are crossed, and if
I feel that 'my chair' or 'my-body-on-the-chair' is being monitored,
this can be perceived as intrusive. 

O'Hara's \cite{Ohara-2016-TSVP} approach to privacy involves seven
`levels', of which the first level is occupied by the \term{underlying
  concept of privacy} --- provisionally identified in terms of whether a
signifcant boundary has been crossed. The second privacy level concerns the \term{empirical
  facts} of the matter. The monitoring project took steps to ensure
that no personal data was collected by the sensors; in particular, the
chair sensors were configured in such a way that they only allow binary
discrimination, namely was a `sitting' event detected at a given
measuring period or not.

 O'Hara's third level is characterised in terms
of \term{phenomenology}: regardless of the empirical facts of privacy, how is the
situation perceived by the subject. In the case of, say, a social media
platform, I may feel that I am having a private conversation with a
friend, unaware that in fact a lot of information is being collected
by the company that owns the platform. However, if the presence of IoT
monitoring devices are explicitly signalled to a user, and indeed have
a visible form factor, then the converse perception of being
surveilled may easily arise, event if the perception is ill-founded.

In terms of deploying an IoT monitoring system, we are therefore
confronted by the problem of a potential discrepancy between privacy
level two and level three: we are not collecting personal data, but
the physical devices being used may prompt concerns from people that
the opposite is true. Ideally, we would like to be able to make demonstrably
evident that the data flow from sensor to processing system carries no
personal information, for example by a suitably configured
visualisation in one of the office's public spaces. This is essentially the notion of
\term{computational accountability} discussed in
\cite{Crabtree-2016-BAIT}: ``the surfacing or making visible of
computational behaviours or actions to better enable human-computer
interactions. However, in our case, we are particularly interested in
the ramifications of accountability in the group setting of an office,
rather than in terms of individuals as data subjects.


% \subsection{Purpose and Values}
% \label{sec:communication-values}

As pointed out in Section~\ref{sec:introduction}, most discussions of IoT
systems 
focus on technology stacks
and engineering concerns \cite{Puschel-2016-WIAS,Heidt-2016-PGFT} , with human actors being relegated to the
margins \cite{Shin-2014-ASTF}.
Considering the ‘massive and pervasive’ impact and implications of
IoT, careful and thorough consideration and integration of the
end-user perspective and experience is
necessary \cite{Ziegler-2017-CIOT}.  However, to begin considering the
human adds a layer of complexity and difficulty and requires the
system to become more flexible and adaptive to changing needs and
interests as well as diverse applications \cite{Shin-2017-UTIO}.
While there were commonalities across study participants in terms of
building issues that affected them, it would not have been possible to
address each person’s concerns. Concerns were also described in
greater detail than ideas about how to solve them.

The use of design principles and practices in the context of living
lab experiments can bring users to the forefront of the process and
allow them to ‘co-evolve’ with the technology \cite{Shin-2017-UTIO}.
In this way they not only contribute to the development of the
technology and its uses but also allow themselves to learn about the
new opportunities that it offers and to develop their perspectives and
ideas alongside it.

The issue at stake is not the social acceptability of IoT for the
purpose of getting people to accept it and allowing a technocratic
vision to advance unimpeded. Rather it is how to stimulate curiosity
and ‘educate’ users about technology through active participation ---
in designing the system to meet their needs but also in becoming more
consciously aware of its implications and making informed decisions
about how much they let it shape their lives --- and how much they
will exert their own initiative to shape it.

\subsection{Achieving acceptance}
\label{sec:achieving-acceptance}

We propose three principles for promoting acceptability and trust. The
potential of IoT to maximise resource efficiency and improve user
experience is widely espoused and offers a seemingly easy ‘quick win’
application. However, the way in which IoT applications are designed
and delivered is crucial to their acceptability and either limits or
expands their potential to create shared value. These three guidelines
can help to overcome skepticism and ensure that IoT monitoring
projects incorporate a broader set of values and beneficiaries.

% \paragraph{Transparency} Anyone affected by the monitoring should be
% informed about what data is being collected and should be able to
% access an easily interpretable explanation of this, for example
% through data visualisation. Physical signage in monitored areas is
% more effective than email communication. If possible, some version of
% the raw data should be accessible to anyone interested in exploring it
% and comparing their interpretations to that of the decision-makers. 

% \paragraph{Purpose} Monitoring should be a time-bounded activity with
% a clearly defined purpose. Individuals who might be affected by the
% monitoring should be informed about the what, why and how of decisions
% made based on the data collected. Overmonitoring by “well-meaning
% technophiles” should be avoided when simpler interventions could
% achieve the desired goal. 

% \paragraph{Participation} In the case of workplace monitoring, there
% is an opportunity and a need to move beyond the narratives of resource
% efficiency, maximising productivity and incentivising specific
% behaviours. Trust in management decision-making has been compromised
% by the use of data to justify cost-cutting and provide only a minimum
% acceptable level of facilities and resources. 

\begin{description}
\item[Transparency] Anyone affected by the monitoring should be
informed about what data is being collected and should be able to
access an easily interpretable explanation of this, for example
through data visualisation. Physical signage in monitored areas is
more effective than email communication. If possible, some version of
the raw data should be accessible to anyone interested in exploring it
and comparing their interpretations to that of the decision-makers. 

\item[Purpose] Monitoring should be a time-bounded activity with
a clearly defined purpose. Individuals who might be affected by the
monitoring should be informed about the what, why and how of decisions
made based on the data collected. Overmonitoring by “well-meaning
technophiles” should be avoided when simpler interventions could
achieve the desired goal. 

\item[Participation] In the case of workplace monitoring, there
is an opportunity and a need to move beyond the narratives of resource
efficiency, maximising productivity and incentivising specific
behaviours. Trust in management decision-making has been compromised
by the use of data to justify cost-cutting and provide only a minimum
acceptable level of facilities and resources. 
\end{description}



New narratives built around trust and valuing individuals can be
created by involving employees in identifying the issues that most
affect their performance, comfort, health and well-being and
determining how and whether monitoring could be used to address these
issues effectively. The engagement process can be structured to offer
employees freedom, creativity, and a sense of agency. 


%%% Local Variables: ***
%%% mode: latex ***
%%% coding: utf-8 ***
%%% TeX-engine: xetex ***
%%% TeX-master: "main.tex"  ***
%%% End: ***
