We have described how staff space in a shared commercial meeting was
monitored, focussing particularly on how occupancy levels in a small
meeting room were determined using a variety of connected sensors. The
monitoring project served two goals, both to test the viability of an
end-to-end IoT architecture, and to gather data that would be of
interest to building managers. This allowed us to experiment with a
cluster of issues that were affecting a significant group of people in
their workplace, and to do so in a context where people would generally
tend not to have adverse reactions but rather to be interested in the
project.

The main focus of the work presented here investigates how staff
reacted to having their workplace monitored in this way.  We
conducted a small qualitative study into how they
responded to the use of connected sensors in their environment and what lessons we could
carry forward to improve the design of
future projects along such lines. Based on our analysis of the results
of the investigation, we proposed three
guidelines for achieving user acceptance of IoT workplace deployments,
divided into the categories of transparency, purpose and participation.

We have also argued that projects of this kind could unlock much more
value by going beyond questions such as ``How can I make this
acceptable to staff?” or ``How can I involve users?” By encouraging
users to participate as key stakeholders in the co-design of IoT
experiments, project owners can gain access to pathways, opportunities and ideas that they
would be unlikely to discover by themselves. At the same
time, it can empower people to innovate and discover new possibilities
for using IoT for social benefit.

%%% Local Variables: ***
%%% mode: latex ***
%%% coding: utf-8 ***
%%% TeX-engine: xetex ***
%%% TeX-master: "main.tex"  ***
%%% End: ***