Reflection on design

The original experiment was designed as a technological experiment looking for a useful application. The University was developing an IoT network and needed to look for an opportunity to test and explore the capabilities of the network in a way that would be meaningful and valuable to people. Considering the parameters of the situation, it was a well-informed choice to experiment with an identified problem that was affecting a significant group of people in their workplace, and to do this in a place (Information Services) where people would generally tend not to have adverse reactions and to be interested in the project.

The original purpose of the perceptions survey was to explore how people perceived this project and what we could learn from the project to improve design of future projects.

The principles that we identified will be valuable for future projects…

But in the process we found a deeper line of questioning and a greater opportunity to explore… We still recommend purpose / transparency / participation as guidelines for IoT projects. But we also challenge people to think from the start of an experiment not only, “how can I make this acceptable to users” or “how can I involve users” but also “what possibilities that I hadn’t imagined might arise in the process of this experiment, and how can I be alert to those, open up pathways and opportunities for unimagined possibilities to arise, and both a) empower people to innovate and b) discover new innovations in the process of a ‘standard’ experiment?
