\subsection{Communication}
\label{sec:communication}

One of the first questions we asked was what participants knew about
the monitoring project. Although they had all received the email that
was sent out to all staff in the building, their level of awareness of
the project depended on various factors. Some had not checked their
email and were surprised to find the sensors in the meeting rooms and
in their office space. Others knew more about it because they sat next
to people who were working on the project, were involved in some
aspect of the project themselves, or had a personal interest in the
University IoT programme and were following its activities.

\subsection{Initial Reactions to the Devices}
\label{sec:init-reactions}

A number of people were quite surprised when they discovered the
sensors%
\footnote{ 
Cf.\ the devices illustrated in figure~\ref{fig:poster}.
}
and found them rather curious, referring to them as `Pebbles’
and ‘little guys’ (Interview 2) and describing them as
`strange-coloured objects’ (Interview 2), `pink Stealth bomber’
(Interview 3), `pink lump’, `lump of flesh’ (Focus Group 2) and some
thought if they `were like little animals, like zebras or
leopards...that would be more fun’ (Focus Group 2).

There were some stronger reactions to the monitoring devices that were
placed on seats, referred to as `bums on seats measurers’ (FG 2),
which people felt were fairly intrusive.

\begin{quote}I was surprised when I got down to the room that they were on the
seat part of the chair and you had to physically sit on them. That
makes me uncomfortable.\end{quote} (FG 2)

\begin{quote}It’s just too much in your personal space if it’s in
  physical contact with your rear end.\end{quote} (FG 2)

\begin{quote}It was a bit strange when you went into a meeting room…I wasn’t
expecting to sit on something, that did feel a bit weird. I’ve not
seen anything other than that that bothered me – but the sitting on
that was a bit strange. … It looks like an incontinence pad, it looks
like somebody’s going to have an accident.\end{quote} (FG 2)

\begin{quote}I didn’t like sitting on it either but I probably would have felt a
bit stupid to say I don’t really want to sit on that thing. It’s not
really any different from sitting on a chair…I don’t know, it’s…I
feel…it doesn’t know it’s [my] bum…\end{quote} (FG 2)

\begin{quote}I’ve noticed in that meeting room that I was in, people would choose
to not sit on the chair...they would choose to sit on a different
chair, which I thought was quite interesting…nobody knows it’s you
sitting on the chair, it can’t read your ID card.\end{quote} (SC)

MH: snagging hose…

RG: butt sensors...

Some people at first weren’t quite sure what the devices were doing.
\begin{quote}We were wondering at first what they were...these strange-coloured
objects that were appearing everywhere. [We] guessed it would be
something with monitoring the room occupancy, that they would be
different types of sensors… (CM)\end{quote}

\begin{quote}...it was more than you see that thing and you think, oh, is that
recording what we’re saying?\end{quote} (FG 2) 

However, with the exception of how they felt about the actual
monitoring devices, most people’s general opinion of the occupancy
monitoring project ranged from relatively neutral to enthusiastic and
supportive.

\begin{quote}I think the Internet of Things and the occupancy monitoring is fine,
because actually it’s very anonymous. It’s not being who’s in and out
or who’s part of the meeting or not – it’s just about is it being used
or is it not being used, so I think people are quite relaxed about
that kind of thing. … It was quite clearly explained and because it’s
not on desks which are allocated to fixed named individuals, there’s
no kind of data protection issues or other issues...that I could see\end{quote}
(KL).

\begin{quote}I don’t think they are spying on us. It’s fairly anonymous, and what
you find out from utilisation is what it should be…\end{quote} (MF)

\begin{quote}The whole monitoring thing, I have no problem with it, because it’s
all confidential, and it’s all anonymous\end{quote} (SC)

\begin{quote}I thought it was quite exciting. I’m quite interested in technology
and how technology can be used to deliver services more easily, to
make work life easier, essentially to make things simpler. I think I
was much more on the enthusiastic positive side of it rather than
being negative.\end{quote} (GW)

\subsection{Monitoring the Meeting Room}

As conversation progressed and we got into more of the details of the
project and the issues that people experience in the building, there
were more positive reactions about the potential of the project to
deal with an identified issue. Almost everyone interviewed had
experienced some frustration with the meeting room situation in the
building.

\begin{quote}It will be quite interesting to see what’s produced at the end of it
and see what did we learn about how we use space or do we actually
occupy it when we say we’re meant to...squatter people come in and use
rooms but they don’t actually book them. Often I’ll walk by and these
big rooms will be empty...that’s where a lot of this can help with the
data in terms of what we do actually use it or what we don’t.\end{quote} (CM)

\begin{quote}I think it’s a great idea. I know my perception is…from a
meeting-room perspective, sometimes when you look at a room like this,
there’s two people in it and you think, wait, hang on a minute, I’m
looking for a room with 25 people in it. … It can be quite frustrating
when you’ve got 2 people in a room of 10. Smaller rooms seem to be in
demand a lot more for 1-to-1s, half an hour slots, that kind of
thing…I’ll be interested if the data you’ve been collecting shows that
the smaller rooms are actually in use as private rooms rather than the
open spaces. The [breakout spaces] are not confidential, they’re not
private enough for particular types of conversation, and the noise
carries.\end{quote} (SC)

\begin{quote}With these meeting rooms even though there will be times in the day
when some of the meeting rooms will be empty – not all that much
because it’s very difficult to get a meeting room if you’re organising
a meeting – it’s difficult to get a room basically for a meeting at
less than a fortnight’s notice, and I’m anticipating that will only
get more difficult as more people twig how to check all of the rooms.\end{quote}
(FG 2)

\begin{quote}I’m often booking small meetings in big meeting rooms down here
because you do occasionally need the big rooms, but what we need is
more rooms.\end{quote} (FG 2)

\begin{quote}I’ll book a meeting room which has got capacity of 20 just for me and
one other person because we’ve got nothing else…the occupancy is
really useful for that sort of thing because you can see, are these
large meeting rooms actually being used full all the time or are they
only getting 1 or 2 people.\end{quote} (GW)

There were multiple other building issues apart from meeting rooms
that people mentioned. Noise seemed to be a key one that was also in
part linked to the issue of meeting rooms and not having sufficient
spaces to use outside of the open plan office areas - either for
private conversations or for louder group meetings.

\begin{quote}Are you going to do noise sensors in the offices upstairs by any
chance? When you’ve got these breakout areas, the noise rises and then
comes down, and it might help inform what you can do with the
information you’re collecting. It’s very very quiet, and that’s a
really good environment to work in, but when you get a meeting in one
of these breakout areas or one of these pods, the noise level rises
quite significantly.\end{quote} (SC)

\begin{quote}I think it would be quite good to monitor noise levels as well. This
space is a little bit tight...on my floor we have a mix…quite a lot of
developers and there’s also people like me that do more kind of
project management…the developers can be quite noisy because they’re
doing a lot of this talking through stuff…it might be better to have
everyone who needs quiet space in one area and everyone who needs
noisy space in another area. Sometimes you can have quite a lot of
interruptions, there’s just too much noise, and people are talking
about work, it’s not like they’re just messing around, it’s more like
it’s quite disturbing at times.\end{quote} (GW)

‘User’ input

As mentioned in section~\ref{sec:pilot}, we used a simple data
visualisation to show people what data was being collected. The
visualisation allowed people to understand better exactly what data
was being collected and to offer their own ‘interpretation’ of it. It
also opened up opportunities for them to suggest other ways that the
data could be used to address pertinent building issues.

One thing that multiple people noticed and questioned was the data
collected on chair movement. Collecting data on chairs occupied seemed
to make sense in the context of occupancy monitoring, but it wasn’t
clear what data on chair movement would contribute to the
question. However, at least three people mentioned that the
configuration of rooms was an important issue to consider.

\begin{quote}If you could actually see how the chairs are being moved so you see
whether they get put into a theatre style so many times or whether
they get moved into circles…I don’t know whether just knowing that
they’ve moved is that valuable, without seeing how they’ve been moved
or where they’ve been moved to.\end{quote} (GW)

\begin{quote}Actually if you had the number of chairs moved in a high level of
detail, you’d probably be able to see if someone had gone in and
reorganized the room before everybody else arrives, which would show
if they’re reconfiguring the space...timetabling worr[ies] about how
much time people spend configuring space…how effective the meeting can
be or how effective they teaching can be if you’re having to get
everybody get all the chairs and then get them in a circle and put
them all back before you can leave…\end{quote} (CM)

\begin{quote}I come in here and do a [relaxation] class in this room…you’re
talking about use of space, everything gets shoved to the back of the
room…\end{quote} (SC)

One thought that data from the door opening and closing could provide
some interesting insights and motivation to people to arrive on time
for meetings.

\begin{quote}Ah right ok, so that’s people coming in. Interesting that they don’t
all arrive at the same time. Quite often in the University we are
terrible for starting meetings on time...people drip feed their way in
after a meeting...you’ll start a meeting and then someone will pop in
5, 10, 15 minutes into a meeting which can be quite disruptive,
especially if they’ve missed some key questions or key points.\end{quote} (CM)

Asking for people’s input and engaging them with the data being
collected in the project opened up a wider conversation around other
ways that data could be used. It seemed to draw out the people who had
experience in using data to design services and improve user
experience, business analytics and intelligence and environmental
monitoring, among other things. 


\subsection{Management}
\label{sec:management}

Most people felt concerned about a variety of building issues and
mentioned quite a few problems other than access to meeting rooms. In
their daily experience, problems with the toilets, slow lifts, and the
temperature of office space and of meeting rooms were significant
complaints. In addition, the way the open plan offices were structured
seemed to create many issues, primarily related to noise. People did
not have private spaces to make phone calls or have one-on-one
conversations, and informal meeting spaces in the open plan areas led
to meetings being overheard by everyone, whether they needed to
participate or not, and they found this disruptive to their
work. There were also gender-related issues, such as men not wanting
to be seen leaving early when they had the school run, or menopausal
women needing to be able to have more control over the temperature in
their space.  

In terms of how building issues are managed, there were also a variety
of complaints. Because the University is renting part of the building
from a third party, they have less control over the management of
building issues. Participants mentioned that the way that the
maintenance staff treats issues can be prone to a box-ticking
approach; for example if someone complains about the temperature in a
space, after a few days someone will come, will measure the
temperature at the point where air is coming out of a vent, and then
say that everything is fine and as it should be, without considering
the temperature in the different parts of the room where different
people are working, or how the sun or lack thereof affects the
temperature over the course of the day.

People also mentioned that although there was supposed to be a
committee of staff providing input for the design of the new
Information Services space in Argyle House, the committee was not
given the opportunity to provide significant input before the
move. They were assured that after the move they would be consulted
again, but up to the present time that had not hapened.

Finally, some participants noted that a DIY approach to environmentla
monitoring has already been carried out, where a few building
occupants had designed and deployed
their own devices (based on Raspberry Pi) to measure room temperature and
other aspects of the working environment. 

\subsection{Perceptions of monitoring}
\label{sec:perc-monit}

% There were mixed perceptions of the monitoring project, and we
% identified three archetypal responses, summarised as follows: 

% \begin{enumerate}
% \item \textit{I don't care} --- I'm already being monitored in so many ways
%   that are so much more personal e.g., CCTV in the building, mobile
%   phone apps, etc. 
% \item \textit{Data is great} --- There are so many ways that we could use
%   data like this to improve our work experience. I'd
%   be happy to be monitored in even more personal ways if that would
%   bring benefit to me. 
% \item \textit{Show me the change} --- Is all this monitoring really necessary?
%   Can I see what data you’re collecting? Will the data actually be
%   used to inform a change? How will I know if it is? Might there be
%   easier or simpler ways to achieve the necessary change? 
% \end{enumerate}

% The first archetype was typically either an under- or over-informed
% person who was not likely to react negatively to the monitoring. The
% second archetype was typically a person with previous experience with
% using sensors or data analytics in their work. They were not likely to
% react negatively to the monitoring and were eager and willing to
% contribute ideas and suggestions about more monitoring options and how
% they could be used to improve user experience. The third response was
% the most common and represents people with the most potential to view
% IoT initiatives skeptically or negatively. 
% \todo{
%   Can we get incude a rough percentage of people in each of the three categories?
% }

% The majority of participants who fell in this category were
% nevertheless willing to engage or be engaged.  Management
% relationships and trust For a number of respondents, the key issues
% was the relationship between management and employees and what was
% communicated through that relationship. There was a lot of concern
% that monitoring that started out for one purpose could be exploited
% for another, depending on the type of manager.  It all depends on how
% the data is used – single biggest concern = management (not bothered
% by general environmental monitoring, but individual monitoring e.g. of
% desk use more concerning – not as much concerned about facilities
% managers as bosses) 

%%% Local Variables: ***
%%% mode:latex ***
%%% TeX-master: "main.tex"  ***
%%% End: ***