\subsection{Communication}
\label{sec:communication}

One of the first questions we asked was what participants knew about
the monitoring project. Although they had all received the email that
was sent out to all staff in the building, their level of awareness of
the project depended on various factors. Some had not checked their
email and were surprised to find the sensors in the meeting rooms and
in their office space. Others knew more about it because they sat next
to people who were working on the project, were involved in some
aspect of the project themselves, or had a personal interest in the
University IoT programme and were following its activities.

\subsection{Initial Reactions to the Devices}
\label{sec:init-reactions}

A number of people were quite surprised when they discovered the
sensors%
\footnote{ 
Cf.\ the devices illustrated in figure~\ref{fig:poster}.
}
and found them rather curious, referring to them as `Pebbles’
and ‘little guys’ (Interview 2) and describing them as
`strange-coloured objects’ (Interview 2), `pink Stealth bomber’
(Interview 3), `pink lump’, `lump of flesh’ (Focus Group 2) and some
thought if they `were like little animals, like zebras or
leopards\ldots that would be more fun’ (Focus Group 2).

There were some stronger reactions to the monitoring devices that were
placed on seats, referred to as `bums on seats measurers’ (FG 2),
which people felt were fairly intrusive.

\begin{qt}I was surprised when I got down to the room that they were on the
seat part of the chair and you had to physically sit on them. That
makes me uncomfortable.\end{qt} (FG 2)

\begin{qt}It’s just too much in your personal space if it’s in
  physical contact with your rear end.\end{qt} (FG 2)

\begin{qt}It was a bit strange when you went into a meeting room…I wasn’t
expecting to sit on something, that did feel a bit weird. I’ve not
seen anything other than that that bothered me --– but the sitting on
that was a bit strange\ldots It looks like an incontinence pad, it looks
like somebody’s going to have an accident.\end{qt} (FG 2)

\begin{qt}I didn’t like sitting on it either but I probably would have felt a
bit stupid to say I don’t really want to sit on that thing. It’s not
really any different from sitting on a chair\ldots I don’t know,
it’s\ldots I feel\ldots it doesn’t know it’s [my] bum\ldots\end{qt} (FG 2)

\begin{qt}I’ve noticed in that meeting room that I was in, people would choose
to not sit on the chair\ldots they would choose to sit on a different
chair, which I thought was quite interesting\ldots nobody knows it’s you
sitting on the chair, it can’t read your ID card.\end{qt} (SC)

MH: snagging hose…

RG: butt sensors\ldots

Some people at first weren’t quite sure what the devices were doing.
\begin{qt}We were wondering at first what they were\ldots these strange-coloured
objects that were appearing everywhere. [We] guessed it would be
something with monitoring the room occupancy, that they would be
different types of sensors… (CM)\end{qt}

\begin{qt}\ldots it was more than you see that thing and you think, oh, is that
recording what we’re saying?\end{qt} (FG 2) 

However, with the exception of how they felt about the actual
monitoring devices, most people’s general opinion of the occupancy
monitoring project ranged from relatively neutral to enthusiastic and
supportive.

\begin{qt}I think the Internet of Things and the occupancy monitoring is fine,
because actually it’s very anonymous. It’s not being who’s in and out
or who’s part of the meeting or not – it’s just about is it being used
or is it not being used, so I think people are quite relaxed about
that kind of thing. … It was quite clearly explained and because it’s
not on desks which are allocated to fixed named individuals, there’s
no kind of data protection issues or other issues\ldots that I could see\end{qt}
(KL).

\begin{qt}I don’t think they are spying on us. It’s fairly anonymous, and what
you find out from utilisation is what it should be…\end{qt} (MF)

\begin{qt}The whole monitoring thing, I have no problem with it, because it’s
all confidential, and it’s all anonymous\end{qt} (SC)

\begin{qt}I thought it was quite exciting. I’m quite interested in technology
and how technology can be used to deliver services more easily, to
make work life easier, essentially to make things simpler. I think I
was much more on the enthusiastic positive side of it rather than
being negative.\end{qt} (GW)

\subsection{Monitoring the Meeting Room}

As conversation progressed and we got into more of the details of the
project and the issues that people experience in the building, there
were more positive reactions about the potential of the project to
deal with an identified issue. Almost everyone interviewed had
experienced some frustration with the meeting room situation in the
building.

\begin{qt}It will be quite interesting to see what’s produced at the end of it
and see what did we learn about how we use space or do we actually
occupy it when we say we’re meant to\ldots squatter people come in and use
rooms but they don’t actually book them. Often I’ll walk by and these
big rooms will be empty\ldots that’s where a lot of this can help with the
data in terms of what we do actually use it or what we don’t.\end{qt} (CM)

\begin{qt}I think it’s a great idea. I know my perception is\ldots from a
meeting-room perspective, sometimes when you look at a room like this,
there’s two people in it and you think, wait, hang on a minute, I’m
looking for a room with 25 people in it\ldots It can be quite frustrating
when you’ve got 2 people in a room of 10. Smaller rooms seem to be in
demand a lot more for 1-to-1s, half an hour slots, that kind of
thing\ldots I’ll be interested if the data you’ve been collecting shows that
the smaller rooms are actually in use as private rooms rather than the
open spaces. The [breakout spaces] are not confidential, they’re not
private enough for particular types of conversation, and the noise
carries.\end{qt} (SC)

\begin{qt}With these meeting rooms even though there will be times in the day
when some of the meeting rooms will be empty --- not all that much
because it’s very difficult to get a meeting room if you’re organising
a meeting --- it’s difficult to get a room basically for a meeting at
less than a fortnight’s notice, and I’m anticipating that will only
get more difficult as more people twig how to check all of the rooms.\end{qt}
(FG 2)

\begin{qt}I’m often booking small meetings in big meeting rooms
  down here because you do occasionally need the big rooms, but what
  we need is more rooms.\end{qt} (FG 2)

\begin{qt}I’ll book a meeting room which has got capacity of 20 just for me and
one other person because we’ve got nothing else…the occupancy is
really useful for that sort of thing because you can see, are these
large meeting rooms actually being used full all the time or are they
only getting 1 or 2 people.\end{qt} (GW)

There were multiple other building issues apart from meeting rooms
that people mentioned. Noise seemed to be a key one that was also in
part linked to the issue of meeting rooms and not having sufficient
spaces to use outside of the open plan office areas --- either for
private conversations or for louder group meetings.

\begin{qt}Are you going to do noise sensors in the offices upstairs by any
chance? When you’ve got these breakout areas, the noise rises and then
comes down, and it might help inform what you can do with the
information you’re collecting. It’s very very quiet, and that’s a
really good environment to work in, but when you get a meeting in one
of these breakout areas or one of these pods, the noise level rises
quite significantly.\end{qt} (SC)

\begin{qt}I think it would be quite good to monitor noise levels as well. This
space is a little bit tight\ldots on my floor we have a mix\ldots quite a lot of
developers and there’s also people like me that do more kind of
project management\ldots{}the developers can be quite noisy because they’re
doing a lot of this talking through stuff\ldots{}it might be better to have
everyone who needs quiet space in one area and everyone who needs
noisy space in another area. Sometimes you can have quite a lot of
interruptions, there’s just too much noise, and people are talking
about work, it’s not like they’re just messing around, it’s more like
it’s quite disturbing at times.\end{qt} (GW)

\subsection{`User' input}
\label{sec:user-input}


As mentioned in section~\ref{sec:pilot}, we used a simple data
visualisation to show people what data was being collected. The
visualisation allowed people to understand better exactly what data
was being collected and to offer their own ‘interpretation’ of it. It
also opened up opportunities for them to suggest other ways that the
data could be used to address pertinent building issues.

One thing that multiple people noticed and questioned was the data
collected on chair movement. Collecting data on chairs occupied seemed
to make sense in the context of occupancy monitoring, but it wasn’t
clear what data on chair movement would contribute to the
question. However, at least three people mentioned that the
configuration of rooms was an important issue to consider.

\begin{qt}If you could actually see how the chairs are being moved so you see
whether they get put into a theatre style so many times or whether
they get moved into circles\ldots{}I don’t know whether just knowing that
they’ve moved is that valuable, without seeing how they’ve been moved
or where they’ve been moved to.\end{qt} (GW)

\begin{qt}Actually if you had the number of chairs moved in a high level of
detail, you’d probably be able to see if someone had gone in and
reorganized the room before everybody else arrives, which would show
if they’re reconfiguring the space\ldots timetabling worr[ies] about how
much time people spend configuring space\ldots{}how effective the meeting can
be or how effective they teaching can be if you’re having to get
everybody get all the chairs and then get them in a circle and put
them all back before you can leave\ldots{}\end{qt} (CM)

\begin{qt}I come in here and do a [relaxation] class in this room\ldots{}you’re
talking about use of space, everything gets shoved to the back of the
room\ldots \end{qt} (SC)

One thought that data from the door opening and closing could provide
some interesting insights and motivation to people to arrive on time
for meetings.

\begin{qt}Ah right ok, so that’s people coming in. Interesting that they don’t
all arrive at the same time. Quite often in the University we are
terrible for starting meetings on time\ldots people drip feed their way in
after a meeting\ldots you’ll start a meeting and then someone will pop in
5, 10, 15 minutes into a meeting which can be quite disruptive,
especially if they’ve missed some key questions or key points.\end{qt} (CM)

Asking for people’s input and engaging them with the data being
collected in the project opened up a wider conversation around other
ways that data could be used. It seemed to draw out the people who had
experience in using data to design services and improve user
experience, business analytics and intelligence and environmental
monitoring, among other things. 

\begin{qt}I’m quite interested in generally understanding how students use the
University overall\ldots and why they use particular spaces within the
library over other spaces\ldots I’m also interested in\ldots how do you have an
equitable experience digitally as well as physically and what does
that mean in terms of what you need to provide for students? \ldots  I
suspect there’s some information that will be easier to get through
the Internet of things and other ways of monitoring and getting
that\ldots that will help me understand what students are doing and what
services they need. \ldots  I’m always interested in results and what we can
do to make things better.
\end{qt}
(KL) 

\begin{qt}\ldots to be able to look up some kind of app, to go, that desk is free,
that would be ideal for a lot of people who hot desk. I know upstairs
there are a lot of contractors who come in, and they’re always looking
for somewhere to sit. If the desk was free, and it could be worked in
such a way to inform some central location that the desk was
available\ldots or for two weeks because I’m going on holiday\ldots that would be
something that I think a lot of people would appreciate and benefit
from.
\end{qt}
(SC) 

\begin{qt}I would find that quite useful if you could look at the trends of
what sicknesses are in different areas or if there was any kind of
correlation between temperature and density of staff against absence
levels\ldots if you were tracking me and where I’ve been posted and my
absence record according to the different places that I’ve been
posted, you would see that in Student Systems I was off a lot, so that
would indicate there was something possibly wrong with the room or the
job. It’s that thing that we’re always trying to look for evidence and
we don’t always have quantifiable data that we can use.
\end{qt}
(CM) 

Some people didn’t find any ideas immediately coming to mind, but they
thought it would be valuable to crowd-source ideas.

\begin{qt}You could go to a community and ask for ideas\ldots explain what IoT is
about and what its current capabilities are and ask people if they’ve
got any great ideas and people start to engage at that point. Part of
it is about engaging people, trying to get people involved in the
technology as well, to raise curiosity about it. I’m sure there’s lots
of people with bright ideas who could contribute, come up with ideas.
\end{qt}
(MF) 

There was also a sense that technology initiatives could be more open.

\begin{qt}At the moment it feels like it’s all the techies --- not which it has
to be the case, because it’s new technology, and that’s the way it
always starts, but you know, the University is full of bright people,
I’m sure they could come up with a lot of bright ideas about how the
Internet of Things could be used to improve staff or student
experiences. I’m sure I’ve got two or three ideas myself\ldots I’m not sure what
they were\ldots we were talking about IoT one day and I scribbled a few
things down\ldots I think there’s definitely scope to open up in that way to
get people interested --– is to, have you got an idea for an Internet of
Things\ldots ? That would be quite fun actually.
\end{qt}
(MF) 

\subsection{``People don’t want to be reduced to numbers”}
\label{sec:people-dont-want}


While people were for the most part comfortable with the initial
monitoring project at AH, they had plenty of concerns about expanding
monitoring in the workplace, particularly if it touched on them
personally.

\begin{qt}I think there’s more issues if it were my own desk which had a
monitor attached to it. If people\ldots [are] monitoring how much I’m at my
desk, and doing X Y and Z\ldots that’s where you get into a lot of
interesting discussions about how it’s going to be used, what it’s
trying to establish – if it were trying to establish whether me and
another colleague could share a desk\ldots that might be fine, and I’m quite
up for that, but it would be about the communication back to me as an
individual\ldots whether it’s monitoring me, specifically, at that
particular point in time, that I might feel quite different about it.
\ldots Where [monitoring] is getting more difficult is if it’s
identifiable to them and their behavior, people feel less
comfortable.
\end{qt}
(KL)

\begin{qt}I suppose it’s that balance where people maybe worry about what the
consequences are – if you start monitoring someone’s usage of their
own desk, and they’re only at it 50/60\% of the time – would we opt
for hot desking, because actually we can cram 150 staff into space for
100 because nobody’s actually ever at their desk all the time. \ldots  [or]
if you’re monitoring me, are you going to tell someone that actually I
don’t turn up to my desk until 10 o’clock every day, but then maybe
work till 6\ldots how much extra information would you need to make it
valuable other than just saying the occupancy in general is X without
having any contributing factors to contextualize it, and I think
that’s where it becomes\ldots people would be a bit concerned about how it
looks, how it reflects on them, because people always think these
things are going to be negative, rather than we can learn something
from it.
\end{qt}
(CM) 

\begin{qt}I can see that I would be slightly nervous about it [if my desk was
being monitored] in the beginning, I was like, hang on, I’m being
watched, my boss could use that against me potentially\ldots I know managers
that would do that, I wouldn’t do that against my team, but then again
if I did have concerns, I might refer to it, and I think people would
be nervous of that, well hang on a minute, why weren’t you at your
desk, you’ve not got a meeting in your diary, that kind of thing.
\end{qt}
(SC)

\begin{qt}Seems to be three dimensions to it. One is, if you try to make it
less general and more specific and identify the fact that the person’s
based at a certain place, or the sex of an individual, or the age of
an individual – once you start going down that road, that’s one
dimension. The next is --– will one thing lead to another? So once you
give assent for this, will you suddenly find yourself being monitored
--– even anonymously --– for other reasons and more reasons beyond that? \ldots 
So they say they’re going to monitor the way you use your PC for one
reason – to improve efficiency –-- and then a year down the line, if
they come back and say, well, we will use this data to work out
that\ldots there’s a lot of time you’re not working, or you’re not working
on the right stuff, or you’re on your personal email too much --- all
that stuff, that side of things\ldots And also there’s no time bound --– it
seems to be, once it’s in place, it’s in place for good –-- and no one
ever seems to say it comes to an end at some point\ldots is the data going
to be kept forever, and all that stuff, and can you use that data for
other purposes than what they said?
\end{qt}
(MF)

One participant responded to the first questions about the monitoring, 

\begin{qt}I must confess, it doesn’t bother me at all. We’re getting monitored
every day, everything that we do – I’m quite used to that.\end{qt} 

But later,
when she was asked how she would feel about having her desk monitored,
she became a bit more sceptical. 

\begin{qt}Okay, so that might be slightly
different actually. I would question that I supposed. I probably would
want to understand why. As long as I was told what it was for and that
I was able to genuinely see the data that it was getting from that,
then I probably would be alright. I might say definitely if it
happens.\end{qt} (KW)

\subsection{Making monitoring acceptable}
\label{sec:making-monit-accept}

While participants expressed their concerns, they also frequently
expressed the particular reasons why they were concerned and directly
or indirectly suggested things to be done that could assuage their
concerns. They wanted to know the reason for the monitoring and how
the data was being used.

\begin{qt}[if my personal desk was being monitored] \ldots I think I’d want to know
that it was happening, and I’d want to know why it was happening, and
I would want to know how the data is being used and what use it’s
going to be put to afterwards, and what the rationale behind it was.\end{qt}
(KL)

\begin{qt}\ldots it’s when you talk about improvements --- at the moment it’s all
about collecting data --- but you’re not always sure what the purpose
is.\end{qt} (MF)

\begin{qt}I think with a lot of these things it’s just being confident that you
know what you’re signing up to, that you know what’s going to happen
to your data, and you know that if it’s going to be used in any other
way, that you’re always asked permission\ldots it’s made totally clear to
you so that you understand exactly how your data’s going to be used. I
think you have to be sort of transparent with this sort of stuff, and
I think there probably is quite a lack of trust.\end{qt} (GW)

\begin{qt}I suppose it really depends on the purpose\ldots why\ldots what would be the
purpose in collecting that information\ldots is it to see which desks are
used and which ones could be changed to hot desks and things. I
think\ldots I’d be very happy switching to a hot desk situation if there was
more flexibility about working from home\ldots I’d be very interested in a
more flexible working pattern where you could work in different
locations and not necessarily have a fixed desk.\end{qt} (GW)

They wanted the monitoring actions and the resultant decision-making
processes to be communicated clearly, and they wanted to give people
the option to participate in decision-making.

\begin{qt}The key is always being very open and transparent and really
communicating as best you can.\end{qt} (KL)

\begin{qt}You want to be efficient but you also want to be transparent, because
all you’re going to produce is data which people can use and
statistics. And statistics can be interpreted differently – there
always needs to be a context behind statistics.\end{qt} (MF) 

\begin{qt}If people know when a decision is likely to be made and to see the
outcome of participation, however small it might have been – then
people will stay interested. If it’s some nebulous affair where a
decision will be made ‘at some point’ – people won’t believe in it.\end{qt}
(MF)

\begin{qt}If they were looking into making particular decisions or changing
certain things, it would be nice to say, we’re doing x to this meeting
room, or y to this meeting room, do you agree or not agree. If there
were a number of different options for a particular meeting room, you
could see the information, see which the different options were, and
vote, have input. For me one of the more important things is to say,
we’re putting up all this monitoring stuff\ldots yes, it’s actually going
into decision making based on what we’ve found out. Otherwise why are
we bothering to do this and spend lots of time on this if we then
don’t change things and don’t act on the information?\end{qt} (KL)

They wanted to help others understand what monitoring is doing and
what it achieves.

\begin{qt}[referring to technology-enhanced learning] \ldots We’re not really
trying to profile an individual, what we’re trying to do is see what
impact this building has on people or has on a group of people, and I
think when you explain that to people then it’s usually not too bad.\end{qt}
(CM) 

\begin{qt}\ldots it’s getting away from that perception of it being a judgment –
because that’s quite often what people will think, that when you put
numbers against something they see it as being a judgment and actually
numbers are just trying to show what happens and a reflection of the
activity that goes on, and you can get insight from that that isn’t
necessarily about it being good or bad – it’s just what actually goes
on.\end{qt} (CM)

\begin{qt}I do think actually just informing people of the information that’s
being gathered and the options to help improve things\ldots I think a lot
of the time you just have to show people the benefit, rather than just
say oh, we’re collecting this. A lot of people don’t understand why.\end{qt}
(SC)



% \subsection{Management}
% \label{sec:management}

% Most people felt concerned about a variety of building issues and
% mentioned quite a few problems other than access to meeting rooms. In
% their daily experience, problems with the toilets, slow lifts, and the
% temperature of office space and of meeting rooms were significant
% complaints. In addition, the way the open plan offices were structured
% seemed to create many issues, primarily related to noise. People did
% not have private spaces to make phone calls or have one-on-one
% conversations, and informal meeting spaces in the open plan areas led
% to meetings being overheard by everyone, whether they needed to
% participate or not, and they found this disruptive to their
% work. There were also gender-related issues, such as men not wanting
% to be seen leaving early when they had the school run, or menopausal
% women needing to be able to have more control over the temperature in
% their space.  

% In terms of how building issues are managed, there were also a variety
% of complaints. Because the University is renting part of the building
% from a third party, they have less control over the management of
% building issues. Participants mentioned that the way that the
% maintenance staff treats issues can be prone to a box-ticking
% approach; for example if someone complains about the temperature in a
% space, after a few days someone will come, will measure the
% temperature at the point where air is coming out of a vent, and then
% say that everything is fine and as it should be, without considering
% the temperature in the different parts of the room where different
% people are working, or how the sun or lack thereof affects the
% temperature over the course of the day.

% People also mentioned that although there was supposed to be a
% committee of staff providing input for the design of the new
% Information Services space in Argyle House, the committee was not
% given the opportunity to provide significant input before the
% move. They were assured that after the move they would be consulted
% again, but up to the present time that had not hapened.

% Finally, some participants noted that a DIY approach to environmentla
% monitoring has already been carried out, where a few building
% occupants had designed and deployed
% their own devices (based on Raspberry Pi) to measure room temperature and
% other aspects of the working environment. 

% \subsection{Perceptions of monitoring}
% \label{sec:perc-monit}

% There were mixed perceptions of the monitoring project, and we
% identified three archetypal responses, summarised as follows: 

% \begin{enumerate}
% \item \textit{I don't care} --- I'm already being monitored in so many ways
%   that are so much more personal e.g., CCTV in the building, mobile
%   phone apps, etc. 
% \item \textit{Data is great} --- There are so many ways that we could use
%   data like this to improve our work experience. I'd
%   be happy to be monitored in even more personal ways if that would
%   bring benefit to me. 
% \item \textit{Show me the change} --- Is all this monitoring really necessary?
%   Can I see what data you’re collecting? Will the data actually be
%   used to inform a change? How will I know if it is? Might there be
%   easier or simpler ways to achieve the necessary change? 
% \end{enumerate}

% The first archetype was typically either an under- or over-informed
% person who was not likely to react negatively to the monitoring. The
% second archetype was typically a person with previous experience with
% using sensors or data analytics in their work. They were not likely to
% react negatively to the monitoring and were eager and willing to
% contribute ideas and suggestions about more monitoring options and how
% they could be used to improve user experience. The third response was
% the most common and represents people with the most potential to view
% IoT initiatives skeptically or negatively. 
% \todo{
%   Can we get incude a rough percentage of people in each of the three categories?
% }

% The majority of participants who fell in this category were
% nevertheless willing to engage or be engaged.  Management
% relationships and trust For a number of respondents, the key issues
% was the relationship between management and employees and what was
% communicated through that relationship. There was a lot of concern
% that monitoring that started out for one purpose could be exploited
% for another, depending on the type of manager.  It all depends on how
% the data is used – single biggest concern = management (not bothered
% by general environmental monitoring, but individual monitoring e.g. of
% desk use more concerning – not as much concerned about facilities
% managers as bosses) 

%%% Local Variables: ***
%%% mode:latex ***
%%% TeX-master: "main.tex"  ***
%%% End: ***