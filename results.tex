What we found
Communication issues
Other issues (discomfort, annoyance, etc.)
Perceptions of monitoring: Archetypes
Management relationships

\subsection{Communication issues}
\label{sec:communication}

One of the key things we discovered is that many people felt poorly informed about the nature and purpose of the monitoring project. Although they were all included in the email that was sent out explaining the details of the project, many failed to read it among the deluge of daily email. They noticed the sensors placed in the meeting rooms, and they read the notice there, but when sensors and gateways were placed around their open plan office area, there was open speculation and discussion about why they were there.

\subsection{Management}
\label{sec:management}

Most people felt concerned about a variety of building issues and mentioned quite a few problems other than access to meeting rooms. In their daily experience, problems with the toilets, slow lifts, and the temperature of office space and of meeting rooms were significant complaints. In addition, the way the open plan offices were structured seemed to create many issues, primarily related to noise. People did not have private spaces to make phone calls or have one-on-one conversations, and informal meeting spaces in the open plan areas led to meetings being overheard by everyone, whether they needed to participate or not, and they found this disruptive to their work. There were also gender-related issues, such as men not wanting to be seen leaving early when they had the school run, or menopausal women needing to be able to have more control over the temperature in their space. 

In terms of how building issues are managed, there were also a variety of complaints. Because the University is renting part of the building from a third party, they have less control over the management of building issues. This means that responses are sometimes slower than in University-owned properties. Participants mentioned that the way that the maintenance staff treats issues is rather technical, for example if someone complains about the temperature in a space, after a few days someone will come, they will measure the temperature at the point where air is coming out of a vent, and then will say that everything is fine and as it should be, without considering the temperature in the different parts of the room where different people are working, or how the sun or lack thereof affects the temperature over the course of the day.

People also mentioned that although there was supposed to be a committee of staff providing input for the design of the new IS space in Argyle House, the committee was not given the opportunity to provide significant input before the move. They were assured that after the move they would be consulted again, but up to the present time that had not hapened. 

People also noted that there were already some people who had designed their own devices (based on Raspberry Pi) to monitor temperature and other issues in the building.

\subsection{Perceptions of monitoring}
\label{sec:perc-monit}

There were mixed perceptions of the monitoring project, and when exploring them, three archetypal responses emerged. These can be described as follows:

\begin{enumerate}
\item “I don’t care” – I’m already being monitored in so many ways that are so much more personal e.g. CCTV in the building, mobile phone, etc.
\item “Data is great” – There are so many ways that we could use this data – and so much more – to improve our work experience. I’d be happy to be monitored in even more personal ways if that would bring benefit to me.
\item “Show me the change” – Is all this monitoring really necessary? Can I see what data you’re collecting? Will the data actually be used to inform a change? How will I know if it is? Might there be easier or simpler ways to achieve the necessary change?
\end{enumerate}

The first archetype was typically either an under- or over-informed person who was not likely to react negatively to the monitoring. The second archetype was typically a person with previous experience with using sensors or data analytics in their work. They were not likely to react negatively to the monitoring and were eager and willing to contribute ideas and suggestions about more monitoring options and how they could be used to improve user experience. The third response was the most common and represents people with the most potential to view IoT initiatives skeptically or negatively. 

The majority of participants who fell in this category were nevertheless willing to engage or be engaged.  Management relationships and trust For a number of respondents, the key issues was the relationship between management and employees and what was communicated through that relationship. There was a lot of concern that monitoring that started out for one purpose could be exploited for another, depending on the type of manager.  It all depends on how the data is used – single biggest concern = management (not bothered by general environmental monitoring, but individual monitoring e.g. of desk use more concerning – not as much concerned about facilities managers as bosses)
