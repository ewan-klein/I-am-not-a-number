\paragraph{Notation} I1, I2,\ldots\ refer to individual interview participants
while (FG1, P1) , (FG1, P2), (FG2, P1) etc.\ refer to focus groups participants.

\subsection{Initial Reactions and Privacy Concerns}
\label{sec:init-reactions}

One of the first questions we asked was what participants knew about
the monitoring project. Although they had all received the email that
was sent out to all staff in the building, their level of awareness of
the project depended on various factors. Some had not checked their
email and were surprised to find the sensors in the meeting rooms and
in their office space. Others knew more about the project because they sat next
to people who were working on it, or they were involved in some
aspect of the project themselves, or they had a personal interest in the
University IoT programme and were following its activities.

A number of people were surprised when they discovered the sensors on
the meeting room chairs,%
\footnote{ Cf.\ the devices illustrated in figure~\ref{fig:poster}.  }
and found them rather curious, referring to them as `pebbles’ and
‘little guys’ (I2), `strange-coloured objects’ (I2), `pink Stealth
bomber’ (I3), `pink lump’, `lump of flesh’ (FG2, P2), and a ‘little
pink bug’ (FG2, P4).

In general, most participants seemed to understand that occupancy
monitoring of the meeting room was not collecting personal data.

\begin{qt}I think the Internet of Things and the occupancy monitoring is fine,
because actually it’s very anonymous. It’s not being who’s in and out
or who’s part of the meeting or not – it’s just about is it [the meeting room] being used
or is it not being used. It was quite clearly explained.\end{qt}
(I1).

\begin{qt}I don’t think they are spying on us. It’s fairly anonymous.\end{qt} (I2)

\begin{qt}The whole monitoring thing, I have no problem with it, because it’s
all confidential, and it’s all anonymous.\end{qt} (I4)

\begin{qt}I thought it was quite exciting. I’m quite interested in technology
and how technology can be used to deliver services more easily, to
make work life easier, essentially to make things simpler. I think I
was much more on the enthusiastic positive side of it rather than
being negative.\end{qt} (I5)

\begin{qt}No [concerns] absolutely not. Not worried at all.\end{qt} (I9)

There were more negative reactions to the monitoring devices that were
incorporated into the chair cushions, referred to as `bums on seats measurers’ (FG2, P4),
which people felt were fairly intrusive.

\begin{qt}I was surprised when I got down to the room that they were on the
seat part of the chair and you had to physically sit on them. That
makes me uncomfortable.\end{qt} (FG2, P1)

\begin{qt}It’s just too much in your personal space if it’s in
  physical contact with your rear end.\end{qt} (FG2, P1)

\begin{qt}It was a bit strange\ldots\  I wasn’t
expecting to sit on something, that did feel a bit weird. It looks like an incontinence pad, it looks
like somebody’s going to have an accident.\end{qt} (FG2, P2)

\begin{qt}I didn’t like sitting on it either but I probably would have felt a
bit stupid to say I don’t really want to sit on that thing. It’s not
really any different from sitting on a chair\ldots\  I don’t know,
it’s\ldots\  I feel\ldots\  it doesn’t know it’s [my] bum\ldots\end{qt} (FG2, P3)

% \begin{qt}I’ve noticed in that meeting room, people would choose
% to not sit on the chair\ldots\  they would choose to sit on a different
% chair, which I thought was quite interesting\ldots\  nobody knows it’s you
% sitting on the chair, it can’t read your ID card.\end{qt} (I4)

\begin{qt}The buttock sensors are a little bit intrusive. Sitting on a sensor seems a step too far.\end{qt} (I6)

\begin{qt}It’s a little bit weird to sit down on the seat with the pad on it.\end{qt} (FG1, P1)

Despite the email about the project and the information poster into
the meeting room, some people had concerns about  what the devices
were doing and what information they were collecting. 

\begin{qt}We were wondering at first what they were\ldots\  these strange-coloured
objects that were appearing everywhere. [We] guessed it would be
something with monitoring the room occupancy, that they would be
different types of sensors\ldots \end{qt} (I3)

\begin{qt}You have to wonder, what data is it gathering, because it
  could be gathering much more data than just a simple binary, is
  someone sitting here or not? Body temperature and everything else
  could be gathered potentially\ldots\ that just feels too
  intrusive.\end{qt} (I6)

\begin{qt}There’s lots of discussion about whether it’s measuring how
  warm you are, whether you’re wiggling, whether you’re wriggling…I
  don’t know that we necessarily know exactly what it’s monitoring. Is
  it monitoring whether I’m asleep? Is it monitoring the level of
  anger and frustration in the room by the wiggling and wriggling and
  fidgeting?\end{qt} (FG1, P1)

\begin{qt}\ldots\  you see that thing and you think, oh, is that
recording what we’re saying?\end{qt} (FG2, P1)

\begin{qt}All I saw was pink things being stuck up\ldots\ there was a
  lot of speculation about what they were.\end{qt} (I9) 



\subsection{Data as Evidence}

As conversation progressed toward more of the details of the
project and the issues that people experience in the building, there
were more positive reactions about the potential of the project to
deal with an identified issue. Almost everyone interviewed had
experienced some frustration with the meeting room situation in the
building.

\begin{qt}It will be quite interesting to see what did we learn about how we use space or do we actually
occupy it when we say we’re meant to. Often I’ll walk by and these
big rooms will be empty\ldots\  a lot of this can help with the
data in terms of what do we actually use.\end{qt} (I3)

\begin{qt}I think it’s a great idea. It can be quite frustrating
when you’ve got 2 people in a room of 10. Smaller rooms seem to be in
demand a lot more for 1-to-1s, half an hour slots, that kind of
thing.\ldots\  The [breakout spaces] are not confidential, they’re not
private enough for particular types of conversation, and the noise
carries.\end{qt} (I4)

\begin{qt}Even though there will be times in the day
when some of the meeting rooms will be empty, it’s difficult to get a room basically for a meeting at
less than a fortnight’s notice.\end{qt} (FG2, P1)

\begin{qt}I’m often booking small meetings in big meeting rooms
  down here because you do occasionally need the big rooms, but what
  we need is more rooms.\end{qt} (FG2, P3)

\begin{qt}The occupancy is
really useful\ldots\  because you can see, are these
large meeting rooms actually being used full all the time or are they
only getting 1 or 2 people.\end{qt} (I5)


As mentioned in Section~\ref{sec:methods}, we used a simple data
visualisation to show people what data was being collected. The
visualisation allowed people to understand better exactly what data
was being collected and to offer their own interpretation of it. It
also opened up opportunities for them to suggest other ways that the
data could be used to address pertinent building issues.

One thing that multiple people noticed and questioned was the data
collected on chair movement. Collecting data on chairs occupied seemed
to make sense in the context of occupancy monitoring, but it wasn’t
clear what data on chair movement would contribute to the
question. However, at least three people mentioned that the
configuration of rooms was an important issue to consider.

\begin{qt}\ldots If you could actually see how the chairs are being moved so you see
whether they get put into a theatre style so many times or whether
they get moved into circles.\end{qt} (I5)

\begin{qt}Actually if you had the number of chairs moved in a high level of
detail, you’d probably be able to see if someone had gone in and
reorganized the room before everybody else arrives, which would show
if they’re reconfiguring the space\ldots\  timetabling worr[ies] about how
much time people spend configuring space\ldots\ {}how effective the meeting can
be or how effective they teaching can be if you’re having to get
everybody get all the chairs and then get them in a circle and put
them all back before you can leave\ldots\ {}\end{qt} (I3)

One person thought that data from the door opening and closing could provide
some interesting insights and motivation to people to arrive on time
for meetings.

[looking at the data visualisation] \begin{qt}Ah right ok, so that’s people coming in. Interesting that they don’t
all arrive at the same time. Quite often in the University we are
terrible for starting meetings on time\ldots\  you’ll start a meeting and then someone will pop in
5, 10, 15 minutes into a meeting which can be quite disruptive.\end{qt} (I3)


A number of people mentioned the value of ‘evidence-based decision-making’ rather than ‘guessing’ (FG1, P1).

\begin{qt}We’re always trying to look for evidence and
we don’t always have quantifiable data that we can use.\end{qt} (I3)

\begin{qt}More data should lead us to better decisions as opposed to
  management by ‘I think and I feel’\end{qt} (I6).

\subsection{Potential for IoT on Campus}
\label{sec:potential}

There were multiple other building issues apart from meeting rooms
that people mentioned. Noise seemed to be a key one that was also in
part linked to the issue of meeting rooms and not having sufficient
spaces to use outside of the open plan office areas --- either for
private conversations or for louder group meetings.

\begin{qt}Are you going to do noise sensors in the offices upstairs by any
chance? When you’ve got these breakout areas, the noise rises and then
comes down, and it might help inform what you can do with the
information you’re collecting. \end{qt} (I4)

\begin{qt}I think it would be quite good to monitor noise levels as well. This
space is a little bit tight\ldots\  on my floor. It might be better to have
everyone who needs quiet space in one area and everyone who needs
noisy space in another area. There’s just too much noise, and it’s
quite disturbing at times.\end{qt} (I5)

Asking for people’s input and engaging them with the data being
collected in the project opened up a wider conversation around other
ways that data could be used. It seemed to draw out the people who had
experience in using data to design services and improve user
experience, business analytics and intelligence and environmental
monitoring, among other things.

\begin{qt}I’m quite interested in generally understanding how students
  use the University overall\ldots\ and why they use particular spaces
  within the library over other spaces\ldots\ I’m also interested
  in\ldots\ how do you have an equitable experience digitally as well
  as physically and what does that mean in terms of what you need to
  provide for students? \ldots\ I’m always interested in results and
  what we can do to make things better.\end{qt} (I1)

\begin{qt}\ldots\  to be able to look up some kind of app, to go, that desk is free,
that would be ideal for a lot of people who hot desk. That would be
something that I think a lot of people would appreciate and benefit
from.\end{qt} (I4)

\begin{qt}I would find that quite useful if you could look at the trends of
what sicknesses are in different areas or if there was any kind of
correlation between temperature and density of staff against absence
levels\ldots\  if you were tracking me\ldots\ you would see that in Student Systems I was off a lot, so that
would indicate there was something possibly wrong with the room or the
job.\end{qt} (I3)

\begin{qt}[data from sensors] can help us to test some
  assumptions that we’ve had in the past as well about what we expect
  users to be like. Even if the data simply confirms something that is
  probably quite obvious, it is useful nonetheless to have data to
  prove that. \ldots\ further investment [in] the library needs to be
  evidence-based, and so if we can provide and triangulate data from
  these other sources, I think it would give the University more
  confidence that what we’re doing is a) needed and b) going to
  realise the value to the students.\end{qt} (I7)


% Some people didn’t find any ideas immediately coming to mind, but they
% thought it would be valuable to crowd-source ideas and to engage people, \begin{qt}to explain what IoT is about and what its currently capabilities are\ldots\  to raise curiosity about it.\end{qt} (I2) They also thought that more engagement could counteract the perception that only ‘techies’ were interested in IoT, that outside of the tech community there are a lot of \begin{qt}bright people [who] could come up with a lot of bright ideas about how the
% Internet of Things could be used to improve experiences\ldots\ I’m sure I’ve got two or three ideas myself\ldots\  we were talking about IoT one day and I scribbled a few
% things down\ldots\end{qt} (I2)

\subsection{Surveillance and Management}
\label{sec:people-dont-want}

While people were for the most part comfortable with the existing
scope of the monitoring project, they had concerns about expanding it
within the workplace, particularly if it touched on them
personally. We asked them what they would think about various
scenarios of being personally monitored, from having a monitor on
their desks to having their movement and around the building and
physical activity tracked.

Responses were mixed, with some people being particularly curious
about what could be learned from additional monitoring but most having
some concern about how the data would be used and what the
implications would be.

\begin{qt}If it [a monitor on my desk] were trying to establish whether me and
another colleague could share a desk\ldots\  that might be fine, and I’m quite
up for that, but it would be about the communication back to me as an
individual\ldots\  whether it’s monitoring me, specifically, at that
particular point in time, then I might feel quite different about it.\end{qt} (I1)

\begin{qt}People maybe worry about what the
consequences are –-- if you start monitoring someone’s usage of their
own desk --– would we opt
for hot desking, because actually we can cram 150 staff into space for
100 because nobody’s actually ever at their desk all the time.\end{qt} (I3)

\begin{qt}Seems to be three dimensions to it. One is, if you try to identify the fact that the person’s
based at a certain place, or the sex of an individual, or the age of
an individual – once you start going down that road, that’s one
dimension. The next is --– will one thing lead to another? So once you
give assent for this, will you suddenly find yourself being monitored
--– even anonymously --– for other reasons and more reasons beyond that? \ldots\  
So they say they’re going to monitor the way you use your PC for one
reason – to improve efficiency –-- and then a year down the line, if
they come back and say, well, we will use this data to work out
that\ldots\  there’s a lot of time you’re not working, or you’re not working
on the right stuff, or you’re on your personal email too much\ldots\  And also there’s no time bound --– it
seems to be, once it’s in place, it’s in place for good –-- and no one
ever seems to say it comes to an end at some point\ldots\  is the data going
to be kept forever, and all that stuff, and can you use that data for
other purposes than what they said?\end{qt} (I2)

\begin{qt}If it was staff offices\ldots\  I think colleagues would be
  very disconcerted by that and they would wonder what on earth it
  possibly was that was being monitored and why. We would need to be
  careful that\ldots\  it wasn’t contributing to a culture of
  surveillance.
\end{qt}
(FG1, P1)

One participant responded to the first questions about the monitoring:
 \begin{qt}I must confess, it doesn’t bother me at all. We’re getting monitored
every day, everything that we do --– I’m quite used to that.\end{qt}
But later,
when she was asked how she would feel about having her desk monitored,
she became a bit more sceptical: \begin{qt}Okay, so that might be slightly
different actually. I would question that I supposed. I probably would
want to understand why. As long as I was told what it was for and that
I was able to genuinely see the data that it was getting from that,
then I probably would be alright. I might say definitely if it
happens.\end{qt} (FG1, P2)

There was a particular concern about the personalities and culture of management.

\begin{qt}I can see that I would be slightly nervous about it [if my desk was
being monitored] in the beginning, I was like, hang on, I’m being
watched, my boss could use that against me potentially\ldots\  I know managers
that would do that.\end{qt} (I4)

\begin{qt}One of the places where I worked in the past we had an
  over-the-top clocking in system. Because there was a bad management
  culture there, it got misused. The problem there wasn’t so much that
  we had information. The problem was that there was a bad management
  culture. If your managers know that they want efficiency, and they
  want people to be motivated and productive, the way to do that is
  the human side – that you’ve got to encourage people to be
  efficient, have transparency about your budgets, let people feel
  valued, trust them.\end{qt} (FG2, P1) 

\begin{qt}We might end up in a situation where we’re saying, well only
  97\% of the screens are in use in public labs, so we don’t need to
  make any more space – the fact that the students are all crammed in
  like chickens, well it’s alright, because there are actually 3 desks
  in the library free. So monitoring is good but it depends what the
  management do with it.\end{qt} (FG2, P3) 
 
\begin{qt}Obviously as long as it’s anonymised we’re not very worried,
  but we’re aware of past cases where overzealous management has
  misused technology.\end{qt} (FG2, P4) 

\begin{qt}It could be abused\ldots\  Heavy-handed management which can
  be done with or without technology would be very
  undesirable.\end{qt} (I6) 

\subsection{Transparency and Purpose}
\label{sec:making-monit-accept}

While participants expressed their concerns about the expansion of
monitoring and the collection of personal data, they also directly or
indirectly suggested things to be done that could assuage their
concerns. They wanted to know the reason for the monitoring and how
the data was being used, and whether it would bring any benefit to
them.

\begin{qt}[if my personal desk was being monitored] \ldots\  I think I’d want to know
that it was happening, and I’d want to know why it was happening, and
I would want to know how the data is being used and what use it’s
going to be put to afterwards, and what the rationale behind it was.\end{qt} (I1)

\begin{qt}If you’re going to gather data, only do it if it’s for a
  purpose and you’ve got a real objective.\end{qt} (FG2, P1)
 
\begin{qt}As long as there’s a purpose\ldots\  I think it really helps if you can see how the data is being used…like visualised in some way that is easy to understand and look at whenever you want. I’m okay with sharing some information\ldots\  if there’s some personal benefit to me of providing that data, then I’m usually quite comfortable providing it, but otherwise if I don’t know what they’re going to do with my data then I won’t.\end{qt}(FG1, P2)

\begin{qt}I think with a lot of these things it’s just being confident that you
know what you’re signing up to, that you know what’s going to happen
to your data, and you know that if it’s going to be used in any other
way, that you’re always asked permission\ldots\  it’s made totally clear to
you so that you understand exactly how your data’s going to be used. I
think you have to be sort of transparent with this sort of stuff, and
I think there probably is quite a lack of trust.\end{qt} (I5)

\begin{qt}As long as you know the reasons why something’s being
  collected, that’s really important. As long as it’s very clear what
  the proposed benefits might be as well as what other sorts of
  decisions might be made based on the data, then I don’t think people
  have a problem. I certainly don’t.\end{qt} (I9)

They wanted the monitoring actions and the resultant decision-making
processes to be communicated clearly, and they wanted to give people
the option to participate in decision-making.

\begin{qt}The key is always being very open and transparent and really
  communicating as best you can. If they were looking into making
  particular decisions or changing certain things, it would be nice to
  say, we’re doing x to this meeting room, do you agree or not
  agree. If there were a number of different options for a particular
  meeting room, you could see the information, see which the different
  options were, and vote, have input. For me one of the more important
  things is to say, we’re putting up all this monitoring stuff\ldots\
  yes, it’s actually going into decision making based on what we’ve
  found out. Otherwise why are we bothering to do this and spend lots
  of time on this if we then don’t change things and don’t act on the
  information?\end{qt} (I1)

\begin{qt}If people know when a decision is likely to be made and to see the
outcome of participation, however small it might have been –-- then
people will stay interested. If it’s some nebulous affair where a
decision will be made ‘at some point’ –-- people won’t believe in it.\end{qt}
(I2)

\begin{qt}You want to be efficient but you also want to be transparent, because
all you’re going to produce is data which people can use and
statistics. And statistics can be interpreted differently – there
always needs to be a context behind statistics.\end{qt} (FG2, P3)

\begin{qt}It’s about who has your data and what they’re doing with
  it. Awareness can lead to a very different reaction to what you’d
  expect. People hemorrhage data constantly, but once they start
  thinking about it, you suddenly find they develop this very
  risk-averse behaviour on the back of it. One of the big things about
  collecting data --– you need to create that level of informed
  consent. It’s how you get people engaged and involved, and the
  minute it comes home and it’s to do literally with them, and they
  feel it’s much more personal, you’ll get more out of people.\end{qt}
(I8)

They wanted to help others understand what monitoring is doing and
what it achieves.

\begin{qt}We’re not really
trying to profile an individual, what we’re trying to do is see what
impact this building has on people or has on a group of people, and I
think when you explain that to people then it’s usually not too bad.\end{qt}
(I3)

\begin{qt}\ldots it’s getting away from that perception of it being a judgment –
because that’s quite often what people will think, that when you put
numbers against something they see it as being a judgment and actually
numbers are just trying to show what happens and a reflection of the
activity that goes on, and you can get insight from that that isn’t
necessarily about it being good or bad – it’s just what actually goes
on.\end{qt} (I3)

\begin{qt}I do think actually just informing people of the information that’s
being gathered and the options to help improve things\ldots\  I think a lot
of the time you just have to show people the benefit, rather than just
say oh, we’re collecting this. A lot of people don’t understand why.\end{qt}
(I4)

%%% Local Variables: ***
%%% mode:latex ***
%%% coding: utf-8 ***
%%% TeX-engine: xetex ***
%%% TeX-master: "main.tex"  ***
%%% End: ***
